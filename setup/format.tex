\makeatletter
% pretend to already have loaded beamerfontthememetropolis
\@namedef{ver@beamerfontthememetropolis.sty}{9999/99/99}

% Beamer theme
\usetheme{Xiaoshan}

% 设置字体(参考曾祥东的latex-talk.tex进行设置)
\usefonttheme{serif,professionalfonts}
% 加载字体
\setmainfont{LibertinusSerif}[% 英文字体
  Extension      = .otf,
  UprightFont    = *-Regular,
  BoldFont       = *-Bold,
  ItalicFont     = *-Italic,
  BoldItalicFont = *-BoldItalic,
  Scale          = 1.1]
\setmonofont{Iosevka}[Scale=0.96]% 等宽字体,主要用于代码排版
\setmathfont{LibertinusMath-Regular.otf}% Iosevka数学字体,需要unicode-math支持
\setCJKmainfont{Source Han Serif SC}[ % 中文字体,思源宋体
  UprightFont     = * SemiBold,
  BoldFont        = * Heavy,
  ItalicFont      = * SemiBold,
  BoldItalicFont  = * SemiBold,
  RawFeature      = +fwid]
\setCJKmonofont{Sarasa Mono SC}% 中文等宽字体,Sarasa Mono SC

% Beamer settings
\metroset{progressbar=none}
\setbeamerfont{title}{size=\huge, series=\bfseries}
\setbeamerfont*{subtitle}{size=\large, shape=\itshape}
\setbeamerfont{section title}{size=\Large, series=\bfseries}
\setbeamerfont{frametitle}{size=\large, series=\bfseries}
\setbeamerfont{caption}{size=\footnotesize, series=\bfseries}
\setbeamerfont{footnote}{size=\tiny}
\setbeamerfont{alerted text}{series=\bfseries}
\addtobeamertemplate{institute}{\raggedleft}{}
\setbeamertemplate{title}{%
  \raggedleft
  \linespread{1.0}%
  \inserttitle
  \hspace*{1.2cm}\par
  \vspace*{0.5em}}
\setbeamertemplate{subtitle}{%
  \raggedleft
  \insertsubtitle
  \hspace*{1.2cm}\par
  \vspace*{0.5em}}
\setbeamertemplate{title page}{
  \begin{minipage}[b]{\textwidth}
    \usebeamertemplate*{title graphic}\vfill
    \usebeamertemplate*{title}
    \usebeamertemplate*{subtitle}
    \usebeamertemplate*{title separator}
    \usebeamertemplate*{author}
    \usebeamertemplate*{date}
    \usebeamertemplate*{institute}
    \vfill
  \end{minipage}}
\setbeamertemplate{frame numbering}{{\insertframenumber/\inserttotalframenumber}}%\zhnumber[style=Financial]
\setbeamertemplate{itemize/enumerate subbody begin}{\footnotesize}
\setbeamertemplate{caption}{\parbox{\textwidth}{\centering\insertcaption}\par}
\setbeamertemplate{bibliography item}[text]

% 设置帧标题格式
\setbeamertemplate{frametitle}{
    \ifbeamercolorempty[bg]{frametitle}{}{\nointerlineskip}%
    \@tempdima=\textwidth%
    \advance\@tempdima by\beamer@leftmargin%
    \advance\@tempdima by\beamer@rightmargin%
    %\hspace*{1cm} %%%%%%%%%%%%% For example insert shift to right
    \begin{beamercolorbox}[sep=0.3cm,wd=\the\@tempdima]{frametitle}
        \usebeamerfont{frametitle}%
        \vbox{}\vskip-1ex%
        \if@tempswa\else\csname beamer@ftecenter\endcsname\fi%
        \strut\insertframesubtitle\hfill\strut%\par%
        {%
            \ifx\insertframetitle\@empty%
            \else%
            {\usebeamerfont{frametitle}\usebeamercolor[fg]{frametitle}|\insertframetitle\strut}%\par}%
            \fi
        }%
        \vskip-1ex%
        \if@tempswa\else\vskip-.3cm\fi% set inside beamercolorbox... evil here...
    \end{beamercolorbox}%
  }

% 设置页脚
\setbeamertemplate{footline}
{
    \leavevmode%
    \hbox{%
        \begin{beamercolorbox}[wd=.25\paperwidth,ht=2.25ex,dp=1ex,center]{frametitle}%{author in head/foot}%
            \usebeamerfont{author in head/foot}\insertauthor{}(\insertshortauthor{})
        \end{beamercolorbox}%
        \begin{beamercolorbox}[wd=.5\paperwidth,ht=2.25ex,dp=1ex,center]{frametitle}%{title in head/foot}%
            \usebeamerfont{title in head/foot}\insertshorttitle
        \end{beamercolorbox}%
        \begin{beamercolorbox}[wd=.25\paperwidth,ht=2.25ex,dp=1ex,right]{frametitle}%{date in head/foot}%
            \usebeamerfont{date in head/foot}\insertshortinstitute{}\hspace*{2em}
            \insertframenumber{}/\inserttotalframenumber{}\hspace*{2ex} 
        \end{beamercolorbox}}%
        \vskip0pt%
    }

%改变脚注的符号
\setbeamerfont{footnote}{size=\zihao{7}} % 改变脚注字号

\makeatletter
\def\@fnsymbol#1{\ensuremath{\ifcase#1\or *\or \dagger\or \ddagger\or
   \mathsection\or \mathparagraph\or \|\or **\or \dagger\dagger
   \or \ddagger\ddagger \else\@ctrerr\fi}}
\renewcommand{\thefootnote}{\fnsymbol{footnote}}
\makeatother

% PoZheHao, see https://github.com/CTeX-org/ctex-kit/issues/382
\ExplSyntaxOn
\xeCJK_new_class:n { PoZheHao }
\__xeCJK_save_CJK_class:n { PoZheHao }
\xeCJK_declare_char_class:nn { PoZheHao } { "2014 }
\seq_map_inline:Nn \g__xeCJK_class_seq
  {
    \str_if_eq:nnF {#1} { PoZheHao }
      {
        \xeCJK_copy_inter_class_toks:nnnn { PoZheHao } {#1} { FullRight } {#1}
        \xeCJK_copy_inter_class_toks:nnnn {#1} { PoZheHao } {#1} { FullRight }
      }
  }
\ExplSyntaxOff


%% 自定义相关的名称宏命令
%% ==================================================
%% \newcommand{\yourcommand}[参数个数]{内容}
% 西北农林科技大学各单位名称
\newcommand{\nwsuaf}{西北农林科技大学}
\newcommand{\cie}{信息工程学院}
\newcommand{\cs}{计算机科学系}


%% 签署春秋学期日期命令
\newcommand{\tomonth}{
  \the\year 年\the\month 月
}


\newcommand{\tomonthen}{
  \ifcase\the\month
  \or January%
  \or February%
  \or March%
  \or April%
  \or May%
  \or June%
  \or July%
  \or August%
  \or September%
  \or October%
  \or November%
  \or December%
  \fi, \the\year
}

\newcommand{\tosemester}{
  \the\year 年\ 
  \ifcase\the\month
  \or 秋%
  \or 春%
  \or 春%
  \or 春%
  \or 春%
  \or 春%
  \or 春%
  \or 夏%
  \or 秋%
  \or 秋%
  \or 秋%
  \or 秋%
  \fi 
}

\newcommand{\tosemesteren}{  
  \ifcase\the\month
  \or Autumn%
  \or Spring%
  \or Spring%
  \or Spring%
  \or Spring%
  \or Spring%
  \or Summer%
  \or Autumn%
  \or Autumn%
  \or Autumn%
  \or Autumn%
  \or Autumn%
  \fi, \the\year
}

%%%%%%%%%%%%%%%%%%%%%%%%%%%%%%%%%%%%%%%%%%%%%%%%%%%%%%%%%%%%%%%%%%%%%%
% 图像标注命令
% LaTeX Overlay Generator - Annotated Figures v0.0.2
% Created with http://ff.cx/latex-overlay-generator/
% If this generator saves you time, consider donating 5,- EUR! :-)
%%%%%%%%%%%%%%%%%%%%%%%%%%%%%%%%%%%%%%%%%%%%%%%%%%%%%%%%%%%%%%%%%%%%%%
%                         #1          #2       #3         #4           #5          #6            #7           #8
%\annotatedFigureBox{bottom-left}{top-right}{label}{label-position}{box-color}{label-color}{border-color}{text-color}
\newcommand*\annotatedFigureBoxCustom[8]{\draw[#5,thick,rounded corners] (#1) rectangle (#2);\node at (#4) [fill=#6,thick,shape=circle,draw=#7,inner sep=2pt,font=\sffamily,text=#8] {\textbf{#3}};}
\newcommand*\annotatedFigureBoxLabel[4]{\annotatedFigureBoxCustom{#1}{#2}{#3}{#4}{red}{white}{black}{black}}
\newcommand*\annotatedFigureBox[3]{\draw[#3,thick,rounded corners] (#1) rectangle (#2);}
\newenvironment {annotatedFigure}[1]{\centering\begin{tikzpicture}\node[anchor=south west,inner sep=0] (image) at (0,0) { #1};\begin{scope}[x={(image.south east)},y={(image.north west)}]}{\end{scope}\end{tikzpicture}}
%%%%%%%%%%%%%%%%%%%%%%%%%%%%%%%%%%%%%%%%%%%%%%%%%%%%%%%%%%%%%%%%%%%%%%

\newcommand\CASE[1]{{\addfontfeatures{Letters=Uppercase}#1}}
\newcommand\jatext[1]{{\addCJKfontfeatures{Language=Japanese}#1}}
% 超链接及符号
\newcommand\link[1]{\href{#1}{\ \faExternalLinkSquare*}}

% 插图路径设置
% ==================================================
\graphicspath{{figures/}}%图片所在的目录
% ==================================================

%%% Local Variables: 
%%% mode: latex
%%% TeX-master: "../main.tex"
%%% End: 
